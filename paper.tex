\newcommand{\huter}{\textit{Humanism and Terror}}
\newcommand{\dark}{\textit{Darkness at Noon}}

In his essay \huter{}, Maurice Merleau-Ponty presents an ultimately unconvincing
counterpoint to Arthur Koestler's fictionalization of the Soviet show-trials,
\dark{}. Ponty insists that his essay is not intended as a defense of the Soviet regime's
violent practices under Stalin, but rather to inspect and conceptualize them in Marxist
terms. In Ponty's view, Koestler's novel presents a fundamentally liberal critique of
communist violence, and therefore fails to relate itself to the realities of the Soviet
situation. Ponty's communist frameworks for understanding the novel, however, contain
flaws and inconsistencies which cannot be internally reconciled. As such, external
critique like Koestler's are not only reasonable but necessary. In fact, Ponty's attempt
to reconcile the internal minds of Koestler's characters and their actions with Marxist
philosophy completely misses the point of \dark{}, that interpreting and effecting Marxist
ideology led the Soviets to beliefs and actions irreconcilable with that ideology.

% pick apart various complaints of ponty's

%% rubashov's beliefs aren't acutally marxist
%%% well, duh
Fundamental to Ponty's critique of \dark{} is his equating Rubashov's understanding of
Soviet doctrine with Koestler's understanding of Marxism. Referencing a moment from the
end of the novel, when a defeated and fatalistic Rubashov reflects on the relationship
between the individual and the collective \parencite[77]{koestler}, Ponty writes,
\textquote[{\cite[23]{ponty}}]{who said that history is a clockwork and the individual a
  wheel? It was not Marx; it was Koestler. It is strange that in Koestler there is no
  inkling of the commonplace notion \textelp{} Koestler has never given much thought to
  the simple idea of a dialectic in history}. As a self-proclaimed dramaturg, it is
difficult for me to express my indignation at Ponty's reading without vastly and rudely
departing from the tonal and stylistic conventions of a college essay. In the context of
\dark{}'s third-person limited narration, any halfway-literate reader understands that the
passage quoted by Ponty is a representation of Rubashov's subjective state of mind, not an
expression of Koestler's beliefs. I have tried to excuse this lapse of Ponty's to myself,
wondering if perhaps the series of translations and second languages through which he must
have experienced the novel may have occluded the meaning which is so obvious to me,
whether German, English and French literary conventions may be so different as to have led
Ponty astray, but Koestler goes as far as to introduce the passage,
\textquote[{\cite[77]{koestler}}]{Rubashov wandered through his cell. In old days he would
  have shamefacedly denied himself this sort of childish musing. Now he was not
  ashamed}. Ponty's critique here depends on conflating Koestler's own beliefs with the
\enquote{childish musings} he ascribes to a fictional character. Any high school
literature student can see that these two are not interchangeable. I refuse to believe
that Ponty lacks such basic reading comprehension, and so I must conclude that he is
knowingly projecting onto Koestler a misunderstanding which he can attack, a
misunderstanding that Koestler has not shown any signs of suffering.

%%% rubashov's view of history is not the same as marx's, but it is (or at least may be)
%%% that of the soviet communists

We see this same flaw in Ponty's criticism of Koestler's supposed outlook on
history. Ponty reproduces several passages in which Rubashov characterizes History (with a
capital \enquote{H}) as a matter of absolute truth, a force beyond comprehension, and a
linear march from the darkness of the past to a bright future \parencite[19]{ponty}. There
are many more moments in \dark{} than just the ones Ponty cites; of these, perhaps most
striking is when Rubashov reflects on excommunicating Richard, having told him,
\textquote[{\cite[14--15]{koestler}}]{The Party can never be mistaken. \textelp{} The
Party is the embodiment of the revolutionary idea in history. History knows no scruples
and no hesitation. Inert and unerring, she flows towards her goal. \textelp{} History
knows her way. She makes no mistakes. He who has not absolute faith in History does not
belong in the Party's ranks}. Ponty attacks this conception of history both on the grounds
that it is contrary to Marx's conception of \textit{praxis} \parencite[18]{ponty}, as well
as in its own right as a fragile idea which insulates people from their agency
\parencite[19]{ponty}. In these critiques, Ponty is exactly correct. But, despite managing
at least to attribute these thoughts to Rubashov rather than Koestler, Ponty seems to
believe (in fact, Ponty makes this belief quite explicit) that Koestler perceives history
this way, or that Koestler is claiming that this is a Marxist perception of
history \parencite[20]{ponty}. Neither is the case, or at least, neither is evidenced by
\dark{}. Rather, Koestler's claim is that the Soviets held this conception of history,
that the Soviets misinterpreted Marx this way. Ponty gets so, so close to critically
engaging with this claim, writing, \textquote[{\cite[23--24]{ponty}}]{\textelp{} if the
alternatives of subjectivism and objectivism are resolved in Marx's Marxism, the question
still remains whether this is so in communism as a reality and whether the majority of
Communists believe in incorporating subjectivity}. But Ponty doesn't actually grapple with
the question posed by \dark{}, namely, whether the leaders the U.S.S.R. understand the
humanist ideals upon which their state is supposedly founded and whether they translate
those ideals into moral actions.

In fact, Ponty acknowledges in his preface that the Soviets do not live up to their goals,
writing, \textquote[{\cite[xx]{ponty}}]{In the last ten years in the U.S.S.R. the social
hierarchy has become considerably accentuated. The proletariat plays an insignificant role
in the Party Congresses. \textelp{} There is an increasing difference between what
Communists think and what they write because there is a widening gap between their
intentions and their deeds}. In this moment, Ponty gives up the game: he admits, before
even beginning his essay proper, that Koestler's depiction of Soviet politics is
accurate. Ponty even demonstrates his ability to understand other works which have the
same goal as \dark{}, framing a quotation from Raymond Aron's \textit{Essai sur la théorie
de l'histoire dans l'allemagne contempoiraine la philosophie critique de l'histoire},
\textquote[{\cite[xli]{ponty}}]{\textins{Aron} was able to express a view that was not his
own but which he considered to be at least one of the more serious views without anyone
accusing him of toadying Nazi or Communist power}. And yet, Ponty's essay proper labors
under the assumption that Koestler was writing to portray his own conceptualization of the
philosophy of Marx, rather than, as Ponty himself puts it, \enquote{to express a view that
was not his own.}

%%% "but koestler is putting all these thoughts in his book, even if they're fictional,
%%% doesn't that constitute endorsement?"

%%% no, actually, rubashov is a dick and an idiot, and we're not particularly supposed to
%%% like or agree with him, just to understand who he is and why he thinks the way he
%%% does. the whole point of the book is "how did the soviet union produce people like
%%% rubashov, and how do people like rubashov understand the world?"

%% marxist history, violence in history, violence against violence in history, and
%% violence in pursuit of peace

%%% contradiction between requiring humanist critiques but ignoring the humanism in
%%% koestler's critique

One of Ponty's more infuriating critiques of \dark{} is that Koestler, apparently, is
discussing Soviet politics from a liberal perspective. As Ponty puts it, in order to
\textquote[{\cite[xvii]{ponty}}]{understand the Communist problem, it is necessary to
start by placing the Moscow Trials in the revolutionary \textit{Stimmung} of violence
apart from which they are inconceivable}. (Why O'Neill chose to reproduce the German word
\textit{Stimmung} rather than translate it as \enquote{context} or \enquote{atmosphere} I
do not know.) Setting aside the irony that Ponty insists we engage with communism on its
terms while himself refusing to engage with \dark{} on its, Ponty's critique here is
frustrating because \dark{} represents Koestler's attempt to place the Moscow Trials into
the violent revolutionary atmosphere of the U.S.S.R., and to inhabit the mind of a
protagonist who breathed it. It is for this purpose that we see Rubashov reflecting on
undertaking violent revolutionary acts. When informing Richard that he is to be ejected
from the Party, especially, we experience Rubashov's violence. In the sequence, we watch
Rubashov expertly and routinely evade fascist operatives, and we experience his
frustration that Richard becomes intimidated by the presence of a soldier near their
meeting place, risking their cover \parencite[13]{koestler}. We are shown that, without
Richard's knowledge, another communist in his cell has been passing information to
Rubashov, and to Rubashov this is normal \parencite[13]{koestler}. We see both mens'
impassivity as Richard reports to Rubashov the number of communists who have been arrested
and presumably killed, including Richard's wife, and we learn that this is an expected
hazard of the job \parencite[13]{koestler}. We listen as Rubashov explains how the
leaflets Richard has been deseminating have undermined the Party, and though we as readers
find Richard's message less objectionable than the official version, our sense is that
Rubashov, having lived in the world we are being shown, believes in the necessity of the
Party's actions and messages, and believes that Richard is a genuine threat to the
revolution \parencite[14]{koestler}. Richard's reaction makes clear that what Rubashov is
doing is violent; Richard begs, \textquote[{\cite[16]{koestler}}]{You c-can't throw me to
the wolves, c-comrade}. Rubashov, who believes that his revolutionary work is more
important that Richard's life, shows the man no sympathy, and though he does experience
regrets at the end of the sequence, his final thought on the matter is that
\textquote[{\cite[16]{koestler}}]{the affair with Richard had to be concluded}. In short,
Koestler makes a concerted effort in \dark{} to place us in Ponty's
\enquote{\textit{Stimmung} of violence.}

%%% highlight is questions about violence between "now" and "lenin's day"; compare to
%%% lenin's hanging order

Koestler's Rubashov makes several assumptions about the atmosphere of violence in which he
lives, and Ponty does not question these assumptions. Key among these is the assumption
that the Soviets under Lenin were just, and used violence only where necessary or useful
to advance the Revolution. In this version of history, the needless and excessive violence
which Koestler and Ponty both decry began after Lenin's death and Stalin's rise to
power. Ponty makes this assumption explicit, asking,
\textquote[{\cite[xviii]{ponty}}]{Does the violence in today's communism have the same
sense it had in Lenin's day?} In Koestler, who avoids using the names of Soviet officials
(this choice is interesting and worth investigating, but outside the scope of this essay),
the assumption is not tied directly to Lenin, but still revealed in Rubashov's linking
memories to years. He reminisces, \textquote[{\cite[11]{koestler}}]{It was in the year
1933, during the first months of terror \textelp{} The Party was no longer a political
organization; it was nothing but a thousand-armed and thousand-headed mass of bleeding
flesh}. This timeline of the Party's descent by 1933 lines up quite nicely with Lenin's
death in 1924 and Stalin's rise to absolute power by 1927. I will not accuse Koestler
himself of assuming that Lenin was violent but just whereas Stalin was unjust and violent,
but Rubashov appears to believe that, and Ponty certainly does. This perspective was
widespread among both Soviets and global leftists, but must be questioned. As I'm sure
you've by now guessed, the counterpoint I find most striking to Ponty and Rubashov's ideal
of Lenin is Lenin's 1918 telegraph ordering the hanging of 100 landed farmers. Lenin uses
the word \enquote{kulak} derogatorily to describe the landed farmers who opposed
collectivization, and I'll freely admit that based on context clues I first assumed that
the \textit{kulaki} were an ethnic group. In fact, the \textit{kulaki} were a sort of
middle class in tsarist Russia, though not the city-dwelling petty bourgeoisie described
by Marx. Lenin's descriptions of the \textit{kulaki} are absurdly violent; he called them,
\textquote[{\cite[65]{rubinstein}}]{bloodsuckers, vampires, plunderers of people and
profiteers, who fatten on famine}. Though the \textit{kulaki} were better off than
\textit{bednjaki}, they were far from rich; they might have owned a couple head of cattle
or a few acres of land \parencite[94]{conquest}. Still, their refusal to give up crops to
the new government was enough for Lenin to order,

\blockquote[{\cite{lenin}}]{The revolt by the five kulak volost's must be suppressed
without mercy. \textelp{} We need to set an example.

\begin{enumerate}
\item You need to hang (hang without fail, so that the public sees) at least 100 notorious
kulaks, the rich, and the bloodsuckers.

\item Publish their names.

\item Take away all of their grain.

\item Execute the hostages -- in accordance with yesterday's telegram.
\end{enumerate}

This needs to be accomplished in such a way, that people for hundreds of miles around will
see, tremble, know and scream out: let's choke and strangle those blood-sucking kulaks.}

I do not live in Ponty's \textit{Stimmung} of violence, and so by his standards I am
unequipped to criticize Lenin's decision to publicly, violently execute dissenters with
the explicit goal of inspiring his followers to further violence. More importantly, I am
disinclined to engage in the sort of historical speculation that follows from asking,
\enquote{what might have happened if Lenin had not responded so violently to the
\textit{kulaki}? Might some of their number been swayed by promises of a better life,
rather than fear?} No use will come of asking these questions, because we can never know
the answers. We do know for certain, though, that in this historical moment, Lenin was
encouraging violence not against imperialists and capitalists living idly in palaces
supported by throngs of malnourished laborers, but against hardworking farmers who were a
bit further from starvation than their neighbors, objecting to a state seizure of land and
food which would have -- and did -- leave them unable to feed themselves or their
families. And we also know, looking back, that Lenin's choices were a step down the path
of a state whose violence did not \textquote[{\cite[xviii]{ponty}}]{recede\textdel{s} with
the approach of man's future}, as Ponty said it should have. This hardly seems to have the
sense Ponty attributes to his idealized Lenin.

%%% stimmung of violence; this is just a cop-out which prevents criticizing violence

%% resolve section
Ponty's insistence that we critique Soviet violence as a divergence between Soviet dogma
and Marxist ideology is flawed for two reasons: firstly, because Koestler's book is a
critique of the violence in Soviet dogma, not the violence in Marxist ideology; and
secondly, because Ponty refuses to criticize the choices which led Soviet revolutionaries
to construct their dogma.
