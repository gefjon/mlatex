Having sat with this question for a bit longer (and reflected on the feedback you gave my
last essay), and considered the plays I did and didn't like this semester, I think:
theater because it can be specific where other art forms must be general. Specific in its
language, specific in its mode, specific in its content. Specific to its place, specific
to its time, specific to its audience. I'd like to talk about specificity and the
exploration it allows in four plays we experienced: \textit{Roma Armee}, \textit{The New
  Gospel}, \textit{Murmel Murmel} and \textit{Black Bismarck}.

My process in writing this essay was to first select the four plays from our class which
most affected me, the four about which I still had thoughts I needed to share, and then to
examine what it was about those plays that affected me. During that process, I settled on
the realization that the plays which moved me were the ones which spoke to my own lived
experience, and the plays which frustrated me were those that spoke to an experience I had
not lived. Having come to that realization, I think I could go back and talk about a
different set of four plays which might make a stronger argument; I might substitute
\textit{TANZ} in place of \textit{Murmel Murmel} or \textit{The New Gospel}. But I've
already written my four reflections, and I've come back to the top to further preface them
like this.

% I'd like to note, before we begin, that throughout the course of this semester I got my
% hands on Julia Jarcho's \textit{Writing and the Modern Stage}, and it's shaped the way I
% think about text, narrative and drama.

Much of my engagement with these plays is going to be in describing the conceptual spaces
they construct and inhabit. I contend that all art, and in fact all communication, is a
process of building a conceptual space and populating it with ideas, or from another
perspective, building a conceptual space which is capable of expressing the ideas you're
interested in. Math classes do this, computer programs do it, romance novels do it, and
plays do it. Each of these works builds upon an existing space specific to the person
consuming it, a space which is informed by expectations and past experiences such as those
with other works in the same medium or genre. For example, the latest Marvel movie can
drop quite easily into the space I bring with me because it was defined by the thirty
Marvel movies I've already seen, whereas a new HBO show must first inform me of the stakes
and the rules because I don't know whether to expect \textit{Curb Your Enthusiasm} or
\textit{Westworld}. The unique thing about theater is that a precise understanding of
one's audience allows a writer or performer to assume more of a shared foundation than is
possible for artists in other media.

\clearpage{}

\paragraph{\textit{Roma Armee}} is both the play that I enjoyed the most out of any we
watched, and the least interesting among those I'll discuss here. By this I mean, it is
the most readily recognizable as a play, albeit not a drama, and it is easy to recognize
both the play's goals and the methods it uses to achieve them. In terms of conceptual
spaces, \textit{Roma Armee} seemed to occupy the already-extant space of bourgeois
political theater, and used it to construct some ideas about how to perceive Romani
people.

Or at least, that was my experience. Based on our discussion in class, I gather that many
of my peers did not find \textit{Roma Armee} as straightforward as I do. On closer
reflection, my specific background prepared me to understand \textit{Roma Armee} more than
my classmates:

\begin{itemize}
\item As a practicing reform Jew with meaningful religious education, I have spent long
  enough studying the Shoah that my understanding goes beyond merely the unexplained
  deaths of Jews, but includes both knowledge of the propaganda which allowed the Nazis to
  kill all those people, and an understanding that Jews were not unique, but were the
  largest among several groups targeted by the Nazis, alongside the Roma. All this to say,
  I'd been taught, at least vaguely, who the Roma are, and what had been done to them.

\item Growing up, I was into superheroes, especially the X-Men. I was familiar with
  Magneto's history (and had spent some time critiquing the idea \enquote{what if the
    Nazi's victims actually were dangerous supernatural beings?}), and I remember being
  quite upset when the MCU incarnation of Wanda and Pietro Maximoff were generic eastern
  Europeans instead of Roma.

\item I spent enough time in Sweden (which is basically upstate Germany) to have a sense
  of how my cultural and historical awareness compared to the norm: especially politically
  active Europeans would have a deeper understanding of the history than me, as would
  directly affected minorities, but most had only a passing familiarity with the idea of
  the Roma.
\end{itemize}

I believe that \textit{Roma Armee} relied on all of this knowledge, and I also believe
that, pretty reliably, every person in the Gorki's audience for that performance would
have had it, although the background that would've given it to them might have been
different. The superhero thing was probably the least understood among the audience, and
it's also the thing most explicitly explained by the show.

\clearpage{}

\paragraph{\textit{The New Gospel}} doesn't fit my vision for theater. There seemed to be
some confusion about how to classify it, so let me say: this was a film. Reading Milo
Rau's manifesto, I can see how he ended up as a filmmaker. The last three of his points
are devoted to making theater transportable to places other than where it was conceived,
and to conceiving plays in places other than where they will be performed. The logical
result of these goals is film.

\begin{enumerate}[label={Command \arabic{*}}, start=8]
\item \label{raucommand:eight} requires that all plays be transportable in a car or
  van. Transporting a film requires only a projector and a screen, or even better, a
  simple digital transfer.

\item requires that plays be conceived in conflict or war zones (or performed there, but
  that is less interesting, and really just an extension of
  \ref{raucommand:eight}). \textit{The New Gospel} shows us that Rau's goal is to portray
  these conflicts to otherwise insulated audiences. Bringing stories from a conflict zone
  and staging them in Berlin does less to break that insulation than bringing people from
  the conflict zone and literally placing them onstage in Berlin, which in turn does less
  than recording images and sounds and recreating them wholesale.

\item extends \ref{raucommand:eight} by requiring that works actually be transported, not
  just that they be transportable. And it is here that Rau loses what I love about the
  theater: his works can no longer laser-target the expectations, experiences and
  preconceptions of an audience full of Berliners, because they also have to be applicable
  to the expectations, experiences and preconceptions of Parisians, Luxembourgers and
  residents of Istanbul (or wherever the Ghent decides to tour).
\end{enumerate}

With \textit{The New Gospel}, Rau set out to break the norms and expectations of theater
to align with his manifesto, and wound up creating a film. A very beautiful film,
certainly, but one that lacked the theater's ability to speak to its audience in their own
language, or to cooperate with its audience in constructing a shared language. We can see
this, in part, because it was the least referential of all the works I'll discuss: the
only cultural artifact referenced was the passions, which has got to be the most
universally known story there is, at least in Europe and the Americas.

\clearpage{}

\paragraph{\textit{Murmel Murmel}} makes this list because I can see that it was specific
to someone's background, but I did not have it. From our discussion, I can see that some
of my classmates did, though, and I am fascinated by the contrast between my experience
and theirs. This play seems to have depended on a framework of movement and physicality
which was present for the dancers and some of the actors in our class, and I suppose for
people with that framework, the things happening onstage communicated ideas in a way
analogous to the words and images in \textit{Roma Armee} or \textit{Black Bismarck} did to
me. For me, though, \textit{Murmel Murmel} was like reading a book in a foreign
language. Or maybe like being given the instruction manual for an IKEA shelf but having to
buy the parts from a Home Depot, and what you end up with just isn't right.

For the purposes of this assignment, I suppose I can let \textit{Murmel Murmel} serve as a
counterpoint, an exemplar of the downside of the hyperspecificity I'm describing. When a
play is for a specific group of people, and it depends on those peoples' foreknowledge,
any outsiders will find it more or less unintelligible. In addition to being better able
to predict its audience relative to film or other media, the theater is also, in my
opinion, more \enquote{allowed,} culturally, to be exclusionary in this way.  I think back
to some discussions I had about the film \textit{The Green Knight}. I loved the movie, but
I think I was only able to love it because I was used to post-dramatic theater, where plot
is not the guiding force. My friends and family who lacked that background did not like
\textit{The Green Knight}, and moreover, they seemed angry that there were movies in
theaters which weren't for them, which depended on a conceptual framework they didn't
have. For all I complain about \textit{Murmel Murmel}, I was certainly not surprised that
I didn't get it; that's just par for the course in the theater.

\clearpage{}

\paragraph{\textit{Black Bismarck},} more than anything else we've seen in this course, is
my vision for post-dramatic theater. I was not within its crosshairs, nor was I quite as
close as I was for \textit{Roma Armee}, but I was certainly much closer than I was for
\textit{Murmel Murmel} or \textit{TANZ}. My language requirement in German left me at
least aware of the linguistic and cultural phenomena the play was responding to.  I
imagine the actual audience, when shown the slideshow of all the monuments to Bismarck,
would have had a realization like, \enquote{huh, I walk past these all the time without
  ever looking too closely or questioning why they were there. I guess I never noticed
  just how prevalent Bismarck monuments are,} but since COVID spoiled my study abroad
plans, I never actually walked past any of the statues. I had seen \textit{Ghostbusters},
but I had to be reminded of the specific context of Stay-Puft's appearance; I had
forgotten about \enquote{clear your mind!}

But after investigating each of these things, I think I can imagine how \textit{Black
  Bismarck} affected its audience, and I am impressed. It seems likely to me that much of
the audience had traveled via that actual subway station, or at the very least one like
it, and had seen the statues that the white rabbit kicked over. A film could not have been
sure of that; filmmakers must contend with the possibility that the same version of
products will be shown in theaters across the country or the world, to people of wildly
different backgrounds, and even in informal settings like living rooms and
airplanes. (Ironically, that's exactly our experience with \textit{Black Bismarck}.)

But (when performed live), \textit{Black Bismarck} could assume those things, and it did,
to great effect. To an actual Berliner, who spoke German as a relatively uncritical
first-language speaker, who experienced the city and its monuments as the background of
everyday life, this play must have masterfully invited (or forced) them to step back and
to critically reexamine their world.

\clearpage{}

For my money, the advantage of theater's specificity is that plays have to do less work
establishing a shared context before they can get to something new and interesting. They
can be a scalpel instead of a chainsaw. If that's not enough for you, then in this space,
feel free to imagine a conclusion about how great theater is which uses the phrases
\enquote{refreshing contrast,} \enquote{intimate,} \enquote{mass-market entertainment,}
\enquote{globalization,} and \enquote{more important than ever.}
