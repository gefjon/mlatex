\newcommand{\lear}{\textit{King Lear}}
\newcommand{\oedipus}{\textit{Oedipus Tyrannus}}
% \section*{Tell 'em what you're gonna tell 'em}
Shakespeare's \lear{} and Sophocles' \oedipus{}, despite being separated by two
millenia and situated in vastly different political systems, show similar
outlooks on the health and fitness of the monarch, of the ruling family, and of
the state. Both plays are concerned with the feedback loop between the stress of
rulership, the monarch's mental health, and the disastrous ripples caused by the
monarch's short-sighted decision-making. Sophocles' more reserved, Aristotelian
dramatic form embodies these concerns almost exclusively in Oedipus himself,
while Shakespeare's greater complexity allows him to spread the embodiments of
illness between Lear and Gloucester. Sophocles' style is more ambiguous,
inviting the audience to empathize with Oedipus even as we watch him descend
into madness and take Thebes with him, whereas Shakespeare is content to have us
dislike Lear, with the tragedy of the play coming not from his personal
suffering but the suffering he inflicts on those around him. Still, the two
plays show remarkable consistency in their depictions of impulsivity,
self-assuredness and rage and their linking of those traits to the suffering and
downfall of the state.

% \section*{Tell 'em}

% \subsection*{Impulsivity, self-assurance, rage}

At the beginning of \oedipus{}, the titular tyrant is quite lucid; it is only as
the play progresses and his fate becomes clear that his mental state is called
into question. He first suffers a burst of rage when Tireseas refuses to reveal
the identity of Laius' killer; Oedipus' inflamed response is,
\textquote[{\cite[ll. 334--335, 339--340]{oedipus}}]{Nothing? Damn you! You
  could make the coldest stone // Burn with rage. \textelp{} Could anyone not be
  angry after hearing // How you hold our city in such contempt?} Though
Tireseas is confoundingly unhelpful, he is not wrong to refuse Oedipus' demands,
nor is Oedipus wrong to be enraged by them. Sophocles shows us that Oedipus, in
the course of his duties as tyrant, is unavoidably exposed to circumstances
which are stressful beyond human tolerance.

Lear, on the other hand, has already succumbed to that stress by the time his
play begins. Instead of watching his descent from lucidity to madness, we are
forced to reconstruct an image of the king he once was by contrasting his older
advisors' loyalty to him against the younger generation's eagerness to oust
him. Kent, after being banished and threatened with death for opposing Lear
\parencite[1.1]{lear}, returns in disguise to continue his service to the
king. Lear is justifiably confused that an apparent stranger would offer himself
into the service of a destitute monarch, and the audience is similarly confused
that Kent would return to his thankless and dangerous work. In an aside, Kent
explains to us, \textquote[{\cite[1.4]{lear}}]{thy master, whom thou lovest, //
  Shall find thee full of labours}. To Lear, Kent gives us a glimpse of the
traits which inspired that love, swearing \textquote[{\cite[1.4]{lear}}]{to
  serve him truly that will put me in trust: to love him that is honest; to
  converse with him that is wise \textelp{}}. A somewhat contrived turn of
dialogue has Lear ask Kent his age, and we are told that Kent is 48. This scene
shows us Lear's former glory, the years Kent spent in service of an honest,
wise, just king. We cannot help but be struck by the contrast between Kent's
description and the Lear who stands before us onstage.

Goneril draws our attention to the transformation again later in the same scene,
entreating her father, \textquote[{\cite[1.4]{lear}}]{I would you would make use
  of that good wisdom \textelp{} and put away // These dispositions, that of
  late transform you // From what you rightly are}. Lear, of course, objects,
insisting that he is now as he's always been, but we know that is not the
case. The stress of rule has taken its toll on Lear, and he is no longer the man
he once was.

The symptoms displayed by Lear and Oedipus bear highlighting: both are prone to
paranoia and fits of rage. Oedipus sees schemes against himself, Thebes and
Laius around every corner; he accuses Tireseas of having part in Laius' death
\parencite[ll. 345--349]{oedipus}, Creon of conspiring to falsely accuse Oedipus
and claim power for himself \parencite[ll. 385--389, 399--402,
572--573]{oedipus}, and espouses a philosophy of rule based in mutual
conspiracy, \textquote[{\cite[ll. 618--621]{oedipus}}]{When a conspirator moves
  quickly against me, // Then I must be quick to conspire back at him. // If I
  hesitate I will lose the initiative, // While he will seize the moment and
  strike}. Sophocles, relentlessly didactic, does not place the onus on us to
interpret and reject Oedipus' paranoia, but has the chorus tell us,
\textquote[{\cite[ll. 616--617]{oedipus}}]{Anyone who treads carefully will
  agree: // Decisions made too quickly are dangerous}. Oedipus then demands that
Creon (who is portrayed in \textit{oedipus} as noble, just and content, with no
lust for power) be put to death, leading to a fast-paced dialogue revelatory of
Oedipus' state:

\begin{minipage}[t]{0.8\linewidth}
  \begin{dialogue}
    \speak{Creon} And you could be wrong.
    \speak{Oedipus} Nevertheless, I must rule.
    \speak{Creon} Not if it means ruling badly.
    \speak{Oedipus} Oh, my city, my city\ldots{}
    \speak{Creon} I belong in this city, it's not only yours!

    \attrib{\cite[ll. 128--130]{oedipus}}
  \end{dialogue}
\end{minipage}

\vspace{0.5\baselineskip}

In this moment, we see that the stresses of ruling (and espeically, of trying to
fix the plague caused by his rule) have driven Oedipus from a just tyranny to
the violence and oppression described by Plato. Oedipus feels pressured to make
quick, firm decisions even when they are wrong, and sees the city as his
possession rather than the object of his service. What is worse, we do not hate
Oedipus for this denigration any more than Kent hates Lear; it is the natural
consequence of circumstances outside Oedipus' control.

% \subsection*{Suffering and downfall of the state}

Both \lear{} and \oedipus{} are keen to show us that, because of their
responsibilities to their respective states, the tragic suffering of the titular
rulers extends to their subjects. For Oedipus, this manifests as a plague, while
for Lear it is a war. (I am tempted to investigate these different portrayals of
danger to the state in search ofsome statement about the political circumstances
of Athens in 429 BCE and England in 1606, but that seems at best tangentially
related to the argument I'm making here.) Again, the structures of these plays
lead us to experience the suffering differently; Sophocles' reliance on a
well-known myth, his limited scope and his character-driven focus places us
already in the plague at the start of the play, without depicting its
beginnings. We are shown, however, the gravity and horror of the plague
\parencite[e.g. ll. 22--30]{oedipus}, and we are reminded that the plague is a
divine punishment for Oedipus' hamartia, his killing of Laius
\parencite[ll. 101--110]{oedipus}. Much later, Sophocles eplicicitly links that
killing to Oedipus' temper \parencite[ll. 807--813]{oedipus}. After, quite
frankly, a very long journey, our understanding of events is: Oedipus, in a
(provoked and arguably justified) fit of rage, killed Laius, then, without
realizing, claimed his throne and house. The gods, outraged that Thebes would,
also without realizing, accept the killer of their rightful king and crown him
tyrant, strike the city with a plague. All of which is to say that the Theban
plague was caused by Oedipus' loss of control over his temper.

% lear causes a war, lmao

At this point in the paper, I had intended to use textal evidence from \lear{}
to convince you that Lear's mental deterioration caused the war between the
English and the French, but honestly, it seems unnecessary. That reading is so
obviously on the surface of this play that I won't waste either of our time
beating a dead horse.

% \subsection*{Blindness}

% \section*{Tell 'em what you told 'em}

Shakespeare's \lear{} and Sophocles' \oedipus{} both present the dangers
inherint in absolute rule: the ruler is placed under incredible stress, and a
ruler who succumbs to stress will cause great harm not just to themself but to
the state and all its citizens. There are, in my mind, two potential responses
to this problem, though neither author explicitly suggests either. In a
monarchist reading, the consequence is that kings (and tyrants) must be
exceptionally steady, level-headed people, and must be surrounded by strong,
trusted advisors who are empowered to stop the ruler from making poor
choices. Both plays show a hurdles in this solution: Lear and Oedipus are both
professed to be the steady, level-headed type, and both are surrounded by
trusted advisors; neither of these safeguards prevents the tragedy of the plays
from unfolding. The antimonarchist reading is, of course, that divesting
absolute rule in a single person is always a losing proposition, no matter how
well-suited that ruler appears, and so political systems which distribute power
among multiple people must be preferred.
