What I propose is not, strictly, a unit of study, but rather, a focus that I would like to
see incorporated into the units already in your class. I find myself fascinated by the
economic realities of performance art, and the ways different systems of funding affect
the works produced. I was struck by this realization during a class from Brandon Woolf and
Andy Goldberg a few weeks ago, when we discussed Berlin's \textit{Freie Szene}. The norm
in Germany is for each theater to permanently employ an ensemble, who are paid a stable
salary by the state. The \textit{Freie Szene}, in contrast, consists of independent
troupes who are funded by government grants on a per-show basis. On the one hand,
\textit{freie} groups exist in a much more precarious space, where a series of rejected
proposals can leave them missing rent or collecting welfare. On the other hand, they enjoy
significant freedoms (hence the name) because of their disconnect from the institutional
hierarchies which employ more traditional ensembles. Notably, it is challenging for a
director working with a salaried ensemble to justify bringing in outside collaborators, so
temporarily hiring subject matter experts to collaborate on particular projects has become
a hallmark of the \textit{Freie Szene}. We got to see this play out onstage: \textit{Roma
  Armee}, a play interested in staging Romani performers, wound up with a cast that was
roughly half Romani, because the Gorki's ensemble only employed a handful of Romani
people. Rimini Protokoll's \textit{100\% Berlin}, on the other hand, invited and placed
onstage 100 Berlin residents. (In truth, I'm not sure if they were paid or volunteers, but
still.) I am interested in applying this same lens to the cultures of performance art we
discussed through the semester.

Throughout this essay, I will perform a relatively shallow investigation of various
economies of performance. I will compare these economies and their results to the
performance economy most familiar to me, Broadway (and American drama more broadly, as I
grew up in Minneapolis), but note that this is merely an artifact of convenience. My goal
is not to privilege Broadway as the optimal performance economy to which all history has
led; in fact, I think it's actually a pretty terrible version, and I find many of the
economies I'll discuss much more compelling.

Also note that most of what I write here is my own analysis based on my background
knowledge and what we discussed in class, along with a non-zero but frankly insufficient
amount of research. I caution you that what I write below will likely not reflect
scholarly consensus, because I'm not sure what the scholarly consensus is.\footnote{I
  blame this at least partially on the focus on novelty and reinterpretation in the
  humanities, which the computer scientist in me is unused to. In my discipline, it's not
  uncommon for the academic community to come to an agreement about the right way to do
  something, and then to stop worrying about it and move on to other things. This does not
  appear to be the case in theater studies or classics.}

\section*{5th century B.C.E. Athens}
% case study: greek drama

%% theater as a public service, performed by soldiers

Our class, like every theater history class, started with the Greeks, and so I suppose I
should, too. My understanding, based largely on a course I took with Peter Meineck to
satisfy my pre-1900s theater requirement, is that Athenian drama, was written by private
citizens for prestige, and performed by enlisted soldiers as part of their duties to the
city. I've also heard it proposed that wealthy Athenians would fund the playwrights, but
this was again with the aim of social prestige, not in exchange for artistic control or
seeking some return on investment. It's difficult to make broad statements about the
effects of this cultural organization on its products, as we have access to only a small
subset of the plays written by the three most successful playwrights, and none by any of
their myriad competitors. Still, we can trace some through-lines.

Athenian dramas tended to be quite specific to the year they were written and staged; I
think of \textit{Ajax} commenting on the rising tensions leading up to the Peloponnesian
war, and of \textit{Oedipus Tyrannus} responding to Pericles' death in the Plague of
Athens. Hyper-specific plays like this are discouraged in many other cultural economies,
including our own, by a desire that scripts be reusable and accessible. A Broadway show
which spoke deeply to its opening-night audience, but within a year started to alienate
all newcomers would be considered a failure, but the Athenian model allowed plays to do
just that.

Being performed as a public service, without a direct financial incentive like ticket
sales, seems to have encouraged Athenian playwrights to focus more on exposing their
audiences to new ideas or perspectives, and less on leaving them entertained or happy,
than the Broadway scene. I think once again of \textit{Ajax}, this time the ambiguous
light in which it portrays Ajax, one of Athens' founding heroes; Odysseus, arguably the
most beloved mythical hero; and Athena, the patron goddess of Athens. I'd love to tell you
what the Greek public thought of \textit{Ajax}, but all I can tell you is that there is no
record of its having won the City Dionysia. We do know that some very challenging and
provocative plays won first place, including Aeschylus' \textit{Persians}, which invited
sympathy for a hated enemy against whom Athens had quite recently fought and won a brutal
war. This suggests to me that the festival's convoluted judging scheme managed to select
for some definition of didacticism and artistic integrity, rather than pure entertainment
value.

\section*{Imperial Rome}

Roman theater, as I understand it, is best summarized by the phrase \enquote{bread and
  circuses.} Roman plays were funded either by the empire itself or by the patricians, and
served (with a range of conscious forethought) to placate and distract the plebeians and
to maintain the social hierarchy. At least as the post-Roman Christians tell it, this
meant the Romans' theater was gaudy, excessive and meaningless. Their main contribution to
drama as an art was in importing, preserving and re-performing the works of other
cultures; every surviving Greek play comes to us through the Romans. If I wanted to be
particularly political, I might set aside some time to draw parallels with modern America,
with the message that empire does not promote compelling art. That essay sounds like a
bummer, though, so instead I'll move on.

\section*{Medieval religious drama}

Frankly, I'm not that interested in this era, which is just as well because there's very
little to say about it. Medieval European history is characterized by the Church's
overwhelming influence over art, culture and day-to-day life. The Romans' excesses led the
Church to reject theater for the better part of a millennium, and when the art does
re-emerge, we see morality plays like \textit{Everyman}. It seems we actually know very
little about \textit{Everyman}; we don't know who wrote it, who first performed it, where
it was performed, or who paid for any of these things. I do feel safe to assert, though,
that the church was directly involved, either in a positive role by sponsoring the play's
creation or a negative one in restricting its content. Comparing this period with Broadway
theater is difficult, as our society is so secular compared to medieval Europe.

\section*{Shakespeare and the renaissance}

It's in the renaissance that we first see a theater economy similar to the one I live with
today, and by Shakespeare's time in the Elizabethan age, the parallels are
remarkable. Playwrights were commissioned by wealthy patrons, especially the monarchs, and
used that funding to support initial performances. They earned profit from ticket sales,
and used the prestige of popular shows to attract patrons for future works. The notable
difference compared to Broadway is that Shakespeare had no obligation to pay back his
patrons' gifts; they were not investors. But the model is fundamentally the same:
monarchs, aristocrats and other people of wealth used financial involvement to dictate the
content of new plays, and ticket sales provided an incentive to produce popular,
crowd-pleasing shows.

Patrons in this era used drama as a means of political conflict and propaganda. On the
world stage, this takes the form of the monarchs commissioning plays which glorify,
historicize and mythologize England's past. We can't actually say for certain which of
Shakespeare's plays were commissioned by Elizabeth or James, or at least, I can't find a
source which justifies the claim, but we do know that a disproportionate number of
Shakespeare's plays are devoted to England's frankly very boring history, and that the
monarchs were his primary patrons. More interesting, on a local level, we see at least one
instance of an internal dispute using a Shakespeare play to justify an attempted coup: the
Earl of Essex famously hired Shakespeare's company, the Chamberlain's Men, to perform
\textit{Richard II} in the hopes of swaying public opinion towards the would-be usurper
against Elizabeth. (It didn't work.)

The popular incentive of ticket sales mostly shows through in Shakespeare's dirty jokes
and reliance on convention. You've already read my take on the dirty jokes, though I'll
point out that they show up not only in \textit{Romeo and Juliet} but also in otherwise
serious tragedies; \textit{Theater Histories} cites the porter scene in
\textit{Macbeth}. In addition to saucy entertainment, audiences expected certain
conventions, especially in terms of plot and style. Comedies had to end with a marriage,
and tragedies with a death; \textit{Theater Histories} here points out that \textit{As You
  Like It} and \textit{Hamlet} had four each of the appropriate ending. Jeremy Lopez has a
whole book about the conventions expected by theater audiences at this time, and the way
playwrights (especially Shakespeare) sought to meet those expectations; he highlights puns
and wordplay, asides and soliloquy, and rapid tonal shifts to keep things exciting. Other
than the specific plot conventions, which have changed to accommodate a new zeitgeist, I
think this whole paragraph could apply to Broadway musicals (dramas tend to be more
experimental and less constrained); they're filled with crowd-pleasing plot moments like
the long-awaited kiss at the end, dirty jokes, songs which illustrate a character's
internal monologue, and pretty much every other device Lopez describes.
