\newcommand{\theplay}{\textit{Romeo and Juliet}}
\newcommand{\billy}{Shakespeare}
\newcommand{\theproduction}{\textit{Slapstick Shakespeare}}

I read \theplay{} in high school, led by an English teacher who frankly wasn't very
good. She taught the play as a serious tragic romance. I experienced it more as a dark
comedy; an elaborate farce at the expense of young people. I was aided by a good annotated
edition (I've unfortunately been unable to find the same one since), which explained all
the dirty jokes.

The plot of \theplay{} always struck me as being... less than romantic. The supposed
\enquote{star-crossed lovers} are a pair of very young teenagers (Juliet is 13) who become
instantly infatuated upon meeting at a party. In fact, it's not even the first time Romeo
has fallen hopelessly in love! I'd like to stage the play in such a way to highlight the
absurdity, bring the raunchy jokes to a forefront, and hopefully destroy the canonical
reading that it's a sincere romance.

The dramatic literature major in me feels obligated to describe my approach as a comedic
\textit{Verfremdungseffekt}. Similar to how Brechtian drama imposes distance in order to
force its audience to critically reconsider the social and political circumstances
presented, \theproduction{} will force its audience to critically reconsider the canonical
interpretation of \theplay{}, particularly in terms of genre and tone.

\section*{A new prologue}

Rather than a chorus, I think I'll read the two prologues (there's one before each of the
first two acts), wearing a mobcap with a feather, to signify my being in-character as the
Bard. Before the first prologue, I'll enter holding the hat, to signify being
out-of-character, and address the audience with an introduction to set the tone:

\begin{dialogue}
  \speak{Phoebe} Hey, y'all. I'm Phoebe, and I'd like to welcome you to this production of
  \theplay{}, \billy{}'s beautiful tragic romance. There will be one 15-minute
  intermission between the second and third acts. Please silence your phones if you
  haven't already, and unwrap your snacks now so they don't crinkle during the show.

  Before we start, I know the language of \billy{} can be hard to get used to, and people
  sometimes have a hard time following, so I'll give you a quick plot summary so you know
  what's going on.

  We start with some exposition, to show you that the houses of Montague and Capulet are
  enemies. It's boring; I'd skip that part if I could. Whatever.

  Romeo of Montague is a thirsty teen, depressed because his crush, Rosaline, doesn't like
  him back. His pals Mercutio and Benvolio sneak him into a party to cheer him up, where
  he meets 13-year-old Juliet of Capulet. They fall in love at first sight. \direct{rolls
    eyes and mimes making themself barf.}

  That night, Romeo sneaks around outside Juliet's house and spies on her a little. He
  hears her talk about how hot he is, so he does his best to act innocent and pretend he
  was just passing by, and reveals himself. They get married the next morning.

  Then, drama! Fiery Tybalt of Capulet kills Mercutio, Romeo kills Tybalt in revenge, and
  the prince comes along and exiles Romeo from the city.

  Juliet decides she doesn't really care that Romeo murdered her cousin, and arranges for
  him to visit that night for sex, before he leaves town forever.

  Then, more drama! Juliet's dad arranges a high-society marriage for her, not knowing
  that Juliet is already married and deflowered. Juliet arranges some convoluted escape
  plan which involves faking her death. Romeo thinks she's actually dead, and immediately
  kills himself out of grief. Juliet wakes up, discovers Romeo's corpse, and kills herself
  for real. Then there's some feel-good stuff where houses Montague and Capulet realize
  the cost of their feud and make peace, which if you ask me brings us dangerously close
  to saying \direct{with airquotes} \enquote{hey kid, kill yourself to solve your parents'
    problems!} But who asked me, right?

  \direct{checks watch.} Oh, shit, I've spent way too long already. Without further ado,
  \direct{dons hat and quickly reads the actual prologue}.
\end{dialogue}

I've also considered breaking this summary up into smaller chunks, which I'd read before
each scene or each act. This could have the added benefits of providing some entertainment
during set changes, and of further distancing the audience by interrupting their
immersion.

\section*{Modifications to the script}

If this staging is to convincingly attack the idea that \billy{} wrote \theplay{} as a
sincere romantic tragedy, it must present his script, unaltered, in its entirety. That
said, many of \billy{}'s best jokes rely on idioms which no longer make any sense. As a
result, a narrator will pause the play at strategic points to explain. (I imagine myself
serving as this narrator, and so I will write my own name in the script.) For example, in
Act 1, Scene 1, supplementing lines 18--24, I would add:

\begin{dialogue}
  \speak{Gregory} The quarrel is between our masters and us their men.

  \speak{Sampson} 'Tis all one, I will show myself a tyrant: when I \\
  have fought with the men, I will be cruel with the \\
  maids, and cut off their heads.

  \speak{Gregory} The heads of the maids?

  \speak{Sampson} Ay, the heads of the maids, or their maidenheads; \\
  take it in what sense thou wilt.

  \speak{Phoebe} \direct{while entering from offstage} Could you pause for a moment, guys?
  
  \direct{to audience} In \billy{}'s time, a \enquote{maiden} is specifically a virgin,
  and when Sampson talks about \enquote{cutting off their maidenheads,} he means fucking
  them.
  
  \direct{beat}
  
  \direct{to \refer{Sampson} and \refer{Gregory}} Thanks, guys, go ahead. \direct{exits}
\end{dialogue}

These interruptions would be kept to a minimum, in order to preserve the flow of the
dialogue. In most cases, strategic emphasis and body language should be sufficient to get
the point across. The lines directly following make for a good example:

\begin{dialogue}
  \speak{Gregory} They \direct{chuckles, then continues while wiggling his eyebrows at the
    audience} must take it in the sense that feel it.

  \speak{Sampson} \direct{mimes holding an exaggerated erect penis} \\
  Me they shall feel while I am able to stand: and \\
  'tis known I am a pretty piece of flesh.

  \speak{Gregory} 'Tis well thou art not if; if thou hadst, thou \\
  hadst been \direct{pauses to produce an ugly wooden fish and hold it to his crotch} poor-John.
\end{dialogue}

\section*{Setting}

One of the most powerful ways to impose distance between the stage and the audience is to
add inconsistencies, so that the play lacks a coherent internal world. This is convenient
because it is easy from a production standpoint, and often reads as comedic.

\theproduction{} will make no effort to ground itself in any one place or time. Visual
motifs in the set and costumes will be chosen with an eye for what best portrays the
correct tone or character. Examples to follow.

\section*{Set}

The set will be dirt-cheap, and will look ridiculous. As with all my choices, my goal here
is to prevent the audience from taking things too seriously.

The only set piece which I see as necessary is the obvious Juliet-style balcony. I haven't
fully decided just how vulgar I want \theproduction{} to be, but I can certainly imagine a
version where the Juliet balcony is shaped and painted to look like a vagina, to emphasize
the sexual subtext in Act 2 Scene 2 and Act 3 Scene 5. A set of pink drapes or blinds
could make this especially funny, with the drapes closed in Act 2 Scene 2 (before the
couple has consummated their marriage), and opened in Act 3 Scene 5 (after the fact). A
wide range of subtler presentations are available, depending on how vulgar I wanted to be;
anything from a simple white wall with pink finish and pink drapes, all the way to an
O'Keeffe painting. I suspect the actual decision would be made by what my investors and
theater administrators would let me get away with.

As to other sets and settings, I imagine:

\begin{itemize}
\item \enquote{A hall in Capulet's house,} where Romeo and friends crash a party and meet
  Juliet, will be styled like a 2000s or 2010s high school dance. There will be a disco
  ball, a punch bowl, and extras in ill-fitting rented suits dancing the \textit{Cha-Cha
    Slide}. The intention is to say: this is where teenagers go for romance and
  excitement.
\item \enquote{Friar Laurence's Cell} will be my best approximation of an authentic monk's
  cell, with a cot, a bench, a candelabra, an ornate cross and little else. The intention
  is to say: here is a man whose intense piety has no place alongside the gaudy wealth and
  teenaged angst of our protagonists.
\item \enquote{A churchyard; in it a tomb belonging to the Capulets} will be decorated
  like a cheap haunted house, with plastic gravestones and fake skeletons purchased at a
  \textit{Spirit Halloween} (or equivalent pop-up store). The intention is to say: this is
  not a serious encounter with morality, it is a morbid punchline.
\end{itemize}

\section*{Costumes}

In no particular order:

\begin{itemize}
\item Prince Escalus and Count Paris, who should appear regal, wealthy and pretentious,
  will wear medieval finery, with frilly shirts, capes, furs and the like. The prince will
  wear a crown and carry a scepter. The count will wear a smaller coronet.
\item Mercutio, who should appear friendly, brash and carefree, will wear jeans and a high
  school letterman jacket. He will carry an open can of beer at all times.
\item Lord Capulet, who should appear a stern businessman who knows that violence is bad
  for profits, will mimic Marlon Brando as the Godfather.
\item Lady Capulet, similarly, will mimic Morgana King as Mama Corleone.
\item Lord Montague will mimic Richard Conte as Emilio Barzini, since I think the
  \textit{Godfather} references are fun.
\item Fiery Tybalt, who's always looking for a fight, will be a greaser, with a leather
  jacket, a cigarette and a shiv.
\item Benvolio, who should appear kind of a wimp, will be a stereotypical nerd. I imagine
  Crispin Glover as George McFly in \textit{Back to the Future}, with thick glasses,
  slicked-back hair, and a dress shirt with the top button buttoned but no tie.
\item Friar Laurence will be a bald monk wearing a brown robe, for similar reasons as my
  design for his cell.
\item Romeo, who should appear angsty, young and stupid, will be a goth teen. Dyed
  jet-black hair, huge steel ear studs, a thin goatee, tight black ripped jeans and a tee
  shirt.
\item Juliet, who should appear young, stupid and naïve, will have long wavy brown hair, a
  pretty blue dress, and lots of blush.
\end{itemize}
