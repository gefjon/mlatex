% tell em what you're gonna tell em

In this paper, I will present a reading Sophocles' \textit{Ajax} through the lens of the
Peloponnesian war. I will justify that lens based on the play's historical context at the
time it was conceived, and based on Sophocles' life and experiences. I will demonstrate
that the reading allows us to understand \textit{Ajax}'s treatment of the goddess Athena,
of militarism and imperialism as a way of life, and of a debate between autocratic and
democratic constitutions.

% why would i think to read Ajax through this lens?

%% yes, i know that ajax predates the war.

In justifying this lens, I must, most importantly, vie with the fact that \textit{Ajax}
was almost certainly written before the Peloponnesian war proper. Though dating Sophocles'
plays is a fraught subject, it is generally accepted that \textit{Ajax} is either the
earliest or among the earliest of the surviving dramas \footcites[\enquote{Dating the
Plays}]{meineck_four}[p. 38]{bates_soph}. Recent estimates place the first performance
of Ajax in the 440s B.C.E. \footcite[p. 10]{finglass_ajax}. At a first glance, this would
place it well before the Peloponnesian war, which is reliably dated to have begun in 431
B.C.E. \footcite[p. 32]{lazenby_war}.

My interest is not, however, in the bloody boots-on-the-ground military conflict to which
Sophocles' later plays respond; rather, in the ongoing tensions between Athenian imperial
expansion and Sparta's more conservative holdings which sparked the war. Thucydides
describes a near-constant conflict between the Athenians and the Spartans beginning almost
immediately after the Persian War \footcite[ch. 1.18]{thucydides_war}. Thucydides, himself
an Athenian who wrote before, during and shortly after the war, is eager to acknowledge
that the war was caused by the Spartans' fear of, and desire to contain, Athens' imperial
expansion  \footcite[ch. 1.23]{thucydides_war}. (An investigation of the social and
political factors which led Thucydides to publish that claim would be fascinating, but
beyond the scope of this paper.)

%% TODO: what the hell is going on in athens at this time?

%% FORESHADOW: sophocles foresaw the war, and ajax is his way of warning his peers of the
%% horrors to come.

%% well, thucydides all but tells me to compare the peloponnesian war w/ the trojan

Thucydides' \textit{History of the Peloponnesian War} also invites this lens by devoting
much of his first book to comparing the Peloponnesian war with the Trojan
war \footcite[e.g. ch. 1.23]{thucydides_war}. If we accept that Sophocles was aware of the
rising tensions between the Athenians and the Spartans when he wrote \textit{Ajax}, then
it is not difficult to imagine him drawing the same parallel as Thucydides between a
contemporary conflict and the Greeks' most historicized mythic war. But we need not merely
imagine, since once one looks through the lens of (pre-) Peloponnesian War Hellenic
geopolitics, a number of otherwise murky features in the text become clear.

% demonstrate the lens: Autocracy and democracy

\textit{Ajax} contains some rather blunt attacks against the coercive nature of autocracy,
and at least one more subtle reference to the voluntary civic engagement required for
democracy. The ideological conflict between autocracy and democracy drove the
Peloponnesian war (or at least, was used to justify it), but categorically could not have
existed during the Trojan War, as democracy was not invented for several centuries.

%% Menelaus

The two least sympathetic characters in \textit{Ajax}, Menelaus and Agamemnon, both
espouse (straw men of) autocracy. Menelaus pronounces, \textquote{Now it is, in truth, the
mark of a base nature when a commoner does not think it right to obey those who stand over
him. Never can the laws maintain a prosperous course in a city where fear has no fixed
place} \footcite[ll. 1072--1075]{ajax}. In this moment, Sophocles has a Spartan advocate
for a rigid social hierarchy without upward mobility, held in place by fear. Menelaus is
also doing this in the same breath as ordering the desecration of the corpse of a
traumatized war hero. Spoken before a crowd of Athenian men, most of them
\enquote{commoners,} and all of them soldiers or veterans, this must have been a powerful
denouncement of autocracy.

%% Aga 1 - irreverent

For Agamemnon, not just a Spartan but their king, Sophocles is even more explicit, having
him say, \textquote{Reverence, I tell you, is not easily practiced by the
autocrat} \footcite[l. 1350]{ajax}. For a second time in \textit{Ajax}, Sophocles tells
us that autocratic rule precludes, or at least devalues, just religious observance. In the
deeply religious society of Hellenic Greece, Sophocles associates autocracy with
godlessness and wickedness, and thus implicitly joins democracy with piety.

%% Aga 2 - shut up & listen

In the same moment, Agamemnon also resists the wisdom of Odysseus, telling him,
\textquote{A good man should listen to those in
charge} \footcite[l. 1352]{ajax}. Odysseus is not entirely sympathetic in \textit{Ajax},
but he is still presented as a paragon of wisdom, piety and morality, at least when
juxtaposed with Agamemnon. And yet Sophocles shows Agamemnon arguing with Odysseus for no
purpose other than personal vendetta, and with no more of an argument than \enquote{you
should listen to me because I am king.} This critique of aristocracy, that powerful rulers
become so rigid and self-assured that they ignore good advice and therefore enact bad
policies, is typical of both classical and modern democratic rhetoric.

%% ajax - a democratic citizen

Sophocles presents Ajax's involvement in the Trojan war as the willing participation of an
outsider in democracy. When defending Ajax against Menelaus' accusations of
insubordination, Teucer says, \textquote{Did he not sail of his own accord? As his own
master? On what grounds are you his commander? \textelp{} Nowhere was it established among
your lawful powers that you should order him any more than he
you} \footcite[ll. 1099--1104]{ajax}. This defense of personal sovereignty, that each
citizen/soldier chose to participate in and contribute to society/the war effort freely,
without duress, and without binding themselves to the will of an individual, could have
come from the textbook of my freshman gen ed course on mixed constitutions. And to
Sophocles, this is what makes Ajax noble: he makes his choices for no reason except that
he thinks they are right, and he collaborates with others to achieve larger goals, even
when personal or ideological differences lead to disagreement.
%% CONSIDER: tie this back in to ``all this doesn't make sense in the actual trojan war,
%% bc a. democracy didn't exist yet, and b. thucydides tells us that agamemnon totally did
%% coerce all his allies''

The ideological conflict between autocracy and democracy justifies this lens, in that the
Peloponnesian war explains easily-recognizable features of the text which make little
sense without the political context of Sophocles' time, but it is not terribly exciting to
learn that a Sophocles play promotes democracy and decries autocracy. To learn anything
new, we must focus our lens on Sophocles' treatment of Athenian militarism and
imperialism, and of the city's god, Athena.

% gain insight from the lens:
% athena - what's her deal?

%% is she the noble goddess of wisdom?

\textit{Ajax} is absolutely rife with references to Athena, who seems to play two somewhat
distinct roles. On the one hand, she is the patron of Odysseus and defender of the Danaans
(who, through this lens, must be taken to represent the Athenians in Sophocles'
time). Odysseus tells us that Athena is \textquote{dearest to \textins{him} of the
  gods} \footcite[l. 14]{ajax}, and later in the play, the chorus praises him for his
wisdom \footcite[ll. 1374--1375]{ajax}, a trait typically associated with Athena. His
Athenian wisdom allows Odysseus to see past the hatred which has blinded Menelaus and
Agamemnon, and to resolve the drama of \textit{Ajax} in an honorable, productive
way.

Athena also tells us that it was her intervention which prevented Ajax from slaughtering
the Greek army, reducing the impact of his crimes to a handful of sheep and herdsmen
\footcite[ll. 40--75]{ajax}. If we read Athena as a personification of the city, we get a
picture of Athens as the imperial protector, using its wisdom to avert conflicts and
catastrophes among its subjects. It's difficult to quantify the extent to which Athens
actually protected its tributaries, as our two main sources, Thucydides and Aristotle,
were both quite critical of this imperial period of Athens' history, showing us a
relationship based more on Athens taking resources from the smaller city-states and using
them to fortify itself \footcite[ch. 23--25]{aristotle_constitution}. But we do know that
the Athenians used defending and educating their allies as a rhetorical justification for
their imperial policies, in much the same way as more recent and current empires. When
shown in a positive light, Sophocles' portrayal of Athena acts as an embodiment of
imperial benevolence, lending wisdom and protection to Odysseus and the Greeks.

%% or is she the god of war?

On the other hand, Athena is also the goddess of war. (Or really, one of many gods of war;
I'm not sure there's any Greek god who wasn't at least a little warlike.) In
\textit{Ajax}, Sophocles portrays her as brash, volatile and cruel. These traits are in
tension with her wisdom and her protection of the Greeks. When Odysseus declines to
confront Ajax at the beginning of the play, Athena admonishes him, \textquote{Hold your
peace! Do not earn a reputation for cowardice!} \footcite[l. 75]{ajax} Odysseus appears to
have made a wise and prudent choice, but Athena encourages him, against his judgment, to
expose himself to unnecessary risk with the goal of proving his bravery. When he pushes
back by explaining the risk, she attempts to entice him, \textquote{is not the sweetest
mockery the mockery of enemies?} \footcite[l. 79]{ajax} Athena's reason for encouraging
Odysseus to do something both tactically and strategically foolish is that he will gain
cruel pleasure from Ajax's suffering.

If we imagine Athena as embodying the city, Odysseus as a noble, educated citizen, and
Ajax as a disenfranchised would-be participant, we can understand Athena's violent urges
in this moment as a commentary on Athens' exclusionary dominance over its tributaries
without democratic representation. Athenian foreign policy at the time Sophocles wrote
\textit{Ajax} consisted largely of using the threat of force to convince a nearby
city-state to convert to democratic rule for internal policies, then to extract labor and
resources from it for the nominal purpose of shared defense. Even more than its actual
military might, this policy depends on Athens appearing unfazed by external threats, and
prepared to punish uppity underlings who won't do as they're told. In his portrayal of
Athena, Sophocles stages this Athenian pattern of coercion via the threat of force, the
attitudes it requires of Athens' persona, and the tensions between those attitudes and the
Athenians' stated values.

%% or is she the bringer of rage and ptsd?

%% TRANSITION: relate this to the aforementioned cruelty.
Athena is directly responsible for Ajax's trauma and madness, and
through them for all the conflict in the play. To the modern eye, she seems to be the
goddess of post-traumatic stress and mental illness, with powers over rage, hallucination
and despair. We see her inspiring rage when she tells Odysseus, \textquote{while
  \textins{Ajax} ran about in diseased frenzy, I kept urging him on, kept hurling him into
  the snares of doom} \footcite[ll. 59--60]{ajax}. By her own admission, Athena stoked the
flames of Ajax's rage, driving him to violence and keeping him going after he would
otherwise have stopped. We see her causing hallucinations when she protects the Achaeans
from that same rage, as she \textquote{prevented \textins{Ajax} by casting over his eyes
  oppressive notions of his fatal joy} \footcite[l. 65]{ajax}. Again by her own admission,
Athena shows us that her method of influencing Ajax is to deceive him, to make him see
things that aren't there.

Most sinister of Athena's powers, though, is that Sophocles attributes to her Ajax's
despair and suicidal ideation. Ajax's depressive episode is called \textquote{the anger of
divine Athena} \footcite[l. 757]{ajax} and \textquote{the intolerable anger of the
goddess} \footcite[l. 778]{ajax}, and we are told that her power is great enough to drive
him to suicide within a day. Ajax recognizes that Athena \textquote{abuses me to my
destruction,} and begs for rest and relief from his despair
\footcite[ll. 400--405]{ajax}. This is a chilling depiction of post-traumatic depression,
and Sophocles has quite explicitly attributed it to Athena, goddess of war and
personification of Athens.

%% she's definitely athens, tho

% is war bad?

Looking back through history, we know that Athens' brashness and cruelty led to a
devastating war, but \textit{Ajax} gives us the sense that Sophocles foresaw the same
thing, or at least feared it. At the very least we can say that \textit{Ajax} is an
anti-war play which criticizes Athenian foreign policy for its aggressive expansion, and
we know that Athens' aggressive expansion led to a calamitous war \footcite[ch. 1.23
ll. 1--2]{thucydides_war}. To the effect of warning the Athenians off their disastrous
path, Sophocles stages the horrors of war in \textit{Ajax}. Ajax's own trauma-induced
suicide is most obvious, but not alone. We also see Odysseus shaped by hyperawareness, a
common symptom of PTSD, in the very first line of the play \footcite[l. 1]{ajax}, showing
that all of the Greek soliders are in some way changed by the war, though some less
spectacularly than Ajax. A description of Ajax's rampage makes a brief reference to the
expected treatment of prisoners of war, as Tecmessa tells us, \textquote{Some
\textins{animals} he beheaded; of some he cut the twisted throat or broke the spine;
others he abused in their bonds as though they were men} \footcite[ll. 295--300]{ajax}.
This whole violent image is gut-wrenching, but the worst part is the implication that
abusing captives is normal. And in case all that were not clear enough, the chorus calls
war \textquote{the \textelp{} plague of Ares' detested arms}
\footcite[ll. 1192--1197]{ajax}. Anti-war imagery and messaged pervade \textit{Ajax}.

%% TODO: war will hurt the peasants, like how athena allows ajax to kill herdsmen

%% and does expansionism cause war?

% tell em what you told em
In attributing to Athena Ajax's crimes, illness and death, Sophocles makes a powerful
claim about the outcome of Athens' imperialist, expansionist and militarist actions. He
showed his audience that Athens' violent behaviors were contrary to its stated goals of
democracy and wisdom, and that continuing down that path would lead to a war which would
traumatize both the Athenians and their imperial subjects. As it turns out, he was right,
and the Peloponnesian war devastated Athens and the Hellenic world not long after
\textit{Ajax}.
% EXPLICATE: what claim?
% athens is fucking people up! good people!
% tie in ajax as democratic participant
