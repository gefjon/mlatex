\newcommand{\theplay}{\textit{Oedipus Tyrannus}}%
\newcommand{\thegame}{\textit{Oedipus Text} (working title)}%
% \section*{Background}

% \subsection*{The source text}

\theplay{} is the most widely-read and widely-known of the Greek tragedies. In
\textit{Poetics}, Aristotle lifts it up as the epitome of the form. \footnote{Why
  Aristotle chooses \theplay{} as the greatest tragedy, why he chooses tragedy as the
  greatest medium, and how \theplay{} compares in form and content to other Greek tragedies
  are all questions for another day; In this project I am concerned only with the fact
  that \theplay{} is, for all intents and purposes, the canonical Greek tragedy.} Because it
is widely taught and widely referenced in later culture and media, \theplay{} presents a
relatively accessible first source text for a new mode of presenting Greek tragedy.

Sophocles' text is quite deep, and adaptors must select from among its themes and facets
what to focus on. In descending order of my interest, I will engage with:

\paragraph*{The tension between destiny and self-determination.} Is the oracle's prophecy
inevitable, or could Oedipus have taken a different path? This is complicated by the
dramatic form; the script written by Sophocles acts as its own prophecy, and no matter how
many times \theplay{} is re-enacted onstage, the outcome will be the same.
  
\paragraph*{Rage inhibiting rational decision-making,} or the tension between Freud's
\textit{it} and \textit{I}. The \textit{hamartia} which sets the tragedy in motion and
ensures the prophecy is Oedipus' loss of self-control when provoked at the
crossroads. Could another man have kept their cool in that moment? This is again
complicated by the nature of the play; Oedipus has to lose control in that moment and kill
Laius, or else there is no story for Sophocles to tell.
  
\paragraph*{Fitness to rule and a monarch's responsibility to their title.} I investigated
this theme in detail in my paper \enquote{Body and the state: Illness in \textit{Oedipus
    Tyrannus} and \textit{King Lear}}, but in short, Oedipus is a good guy whose only flaw
is his tendency to lose control of his rage when provoked. His temper is certainly not a
good thing, but it's fundamentally understandable, and well within the bounds of normal
behavior for a classical Greek nobleman. But still, as \theplay{} shows us, it makes him
unfit to rule Thebes. If a wise, caring, righteous man whose reactions to stressful
circumstances are within social norms is not a fit monarch, is anyone?  \footnote{Spoiler
  for my paper: no, no individual is fit to rule a society.}

% \subsection*{Interactive Fiction}

\paragraph*{Choose-your-own-adventure games} are a genre of Interactive Fiction (IF), or
text-based computer games. Classic IF games like \textit{Zork} used a user-interaction
model where the player typed commands, like \texttt{open door}, \texttt{walk north} or
\texttt{take key}. I feel a certain nostalgia for these games \footnote{Despite having
  been born long after their time\ldots{}}, but they're often frustrating, as the player
has no way of knowing what commands the game will accept. The genre is also known for
oblique puzzles, which I believe tend to distract from the narrative.

More modern narrative text-based games tend instead to use the Choose-Your-Own-Adventure
(CHYOA) model, where relatively long blocks of text are separated by points of
interaction. I will call these points of interaction \enquote{dilemmas,} though that term
is not in the common parlance. At each dilemma, the player chooses between a small number
of options offered by the writer. Subsequent blocks of text then reflect the player's
choice.

When developing CHYOA games, writers are faced with a trade-off between player agency and
implementation effort. On the one hand, the whole point of interactive fiction is to allow
the reader to influence the outcome of the story. Players tend to desire a relative
frequency of dilemmas and a relative abundance of options, and for choices which lead to
meaningfully different outcomes. If a writer's goal was just to tell a singular story that
followed a particular path and ended in a predetermined way, they would eschew the
interactive element and write a traditional narrative in one of the many modes available.

On the other hand, currently extant tools for writing CHYOA games require that the writer
enumerate all possible choices and write the text for all possible outcomes. This is, to
put it lightly, a lot of work. And writers who invest too much effort in offering many
options which lead to diverse outcomes will find that each of those outcomes is less
polished than if they had focused on a smaller number of better-developed paths. Also,
players will often explore only a single path; the prospect of writing one hundred words
of which each reader will experience only ten is creatively unsatisfying.

All narrative computer games grapple with this tension to some extent. \textit{Mass Effect
  3}, the final entry in a high-budget trilogy of role-playing shooter games which
advertised the player's influence over the story, was famously criticized because all
paths converged on similar endings. \textit{Katana Zero}, an independent hack-and-slash
Vietnam War allegory, explored the tension by forcing players to choose between a
\enquote{good} ending which ended the game immediately and a \enquote{bad} ending which
featured significantly more content, at the cost of making the player feel guilty for
actively continuing the game's violence. But the problem is most pronounced in CHYOAs due
to their lack of any gameplay beyond the story. \textit{Black Mirror: Bandersnatch},
Netflix's thoroughly metatextual CHYOA film about the developer of a CHYOA game, is
explicit in its attempt to grapple with the problem.

% \section*{Motivation}

An adaptation of \theplay{} as a CHYOA game provides a unique opportunity to wed
Sophocles' tension between self-determination and fate with the CHYOA writer's tension
between branching paths and finite scope. To that end, I propose \thegame{}. As my first
adaptation of a Greek tragedy into this mode, and, to the best of my knowledge, the first
explicit adaptation of a classical text as a computer game, \thegame{} will necessarily
also serve as a design manifesto.

% \section*{Design}

\newcommand{\selfcontrol}{\texttt{\textsc{Self Control}}}

\newcommand{\knowledge}{\texttt{\textsc{Knowledge}}}

\paragraph*{\thegame{} will be a mystery, where the player attempts to assemble enough
  \knowledge{} to discover the cause of the plague.} As the player traverses the story,
they will uncover facts, which will appear in their inventory. For example, during their
interview with the shepherd, the player will learn \texttt{Laius and Jocasta birthed a
  son, but ordered it put to death by leaving it on a hillside}. Further investigating the
shepherd will add, \texttt{Laius and Jocasta's baby was not put to death, but handed off
  to a Corinthian}. Not all facts will be reliable, and incorrect facts may at times be
replaced with corrected versions; \texttt{I am Oedipus, son of Polybus and Merope, prince
  of Corinth} will eventually be replaced with \texttt{Polybus and Merope adopted me and
  raised me as their son}, and \texttt{Laius was killed by a band of thieves. Plural} with
\texttt{Laius was killed by a lone traveler}.

Completing the game will require the player to accrue enough \knowledge{} to
incontrovertibly prove that Oedipus is the son of Laius and Jocasta and the killer of
Laius.

\paragraph*{The core mechanic of \thegame{} will be \selfcontrol{}, } a finite resource
which can be spent to make better choices. In the first draft of the game, it is
described,

\newcommand{\gamequote}[1]{
  \begin{minipage}[t]{0.8\linewidth}
    \texttt{#1}
  \end{minipage}
  \vspace{\baselineskip}
}

\gamequote{\textsc{Self Control} quantifies your ability to suppress your impulses and
  make good choices. Spend it wisely; fighting your nature and controlling your urges is
  stressful. If you lose your \textsc{Self Control}, you may find yourself making
  impulsive choices you'll end up regretting.}

At each dilemma, some number of options will be available which do not cost any
\selfcontrol{}. These will tend to be more impulsive actions. Some number of options will
also be available which cost various amounts of \selfcontrol{}. These will tend to be more
rational, thought-out actions.

Consider as an example the scene where Creon warns Oedipus to believe Tiresias and not to
act impulsively. \footnote{I have not written this scene yet, so what follows is far from
  final.} The player would then face a dilemma with two available options:

\begin{enumerate}
\item At no cost in \selfcontrol{}, the player could re-enact Sophocles' text by choosing
  to yell at Creon and threaten him with exile.
\item In exchange for one \selfcontrol{}, the player could accept Creon's critique
  graciously and apologize to Tiresias. This would have some yet-undetermined benefit to
  the player, possibly granting them a piece of \knowledge{} which would aid them in
  decoding the mystery.
\end{enumerate}

Crucially, the player's supply of \selfcontrol{} will be constrained such that it is
impossible to reach any ending except the one we know and love from the myth, the poetry
and \theplay{}. Key pieces of \knowledge{}, which, when combined, could lead Oedipus down
a better path, will be strategically locked behind options which, in total, consume more
\selfcontrol{} than the player has access to. This way, players who revisit to \thegame{}
and make different choices in search of a better ending will find that Oedipus' rage
prevents them from doing so.

At key story-defining moments, most notably killing Laius, the player will be presented
with options which cost more \selfcontrol{} than the player's entire reserve. These
moments will always be associated with an injury to Oedipus' feet. The player will
understand that Oedipus' (shame about) his disability

\paragraph*{In order to serve its goals as an exemplar of its new mode,} \thegame{} will
be fiercely and unapologetically metatextual. In addition to narration which will not be
attributed to any speaker, the text will feature design notes in the form of dialogue
attributed to myself, \texttt{phoebe}. In these moments, I will describe to the player my
design choices and intentions, instruct the player as to the game's mechanics and how to
interact with them, and explain background information about the setting. When Oedipus is
crowned Tyrant of Thebes, for example, the player is offered an opportunity to object,
\texttt{Wait, tyrants are bad. I don't want to be a tyrant. Can't I be the King of
  Thebes?} If they do, I respond,

\gamequote{\textsc{phoebe}:
  \begin{minipage}[t]{0.8\linewidth}
    Great question! Tough to answer without a lesson in classical Greek civics, and I
    don't want to get to bogged down in that. But short answer: no, tyrants aren't
    bad. Well, maybe they are, but the Greeks haven't caught on to that yet. In this
    world, a tyrant is someone who takes rulership of a city through might, or through
    cunning, or popular support, or who steps up to lead the people through a crisis. Like
    you did with the Sphinx. A king, on the other hand, is someone who inherits rule from
    their father, or who is appointed by the gods. If you had been the son of Laius, you
    would've inherited kingship of Thebes; but you got the throne in a crisis, so you're a
    tyrant.
\end{minipage}}

From there, if they player has not yet uncovered the truth of their parentage, they may
say, \texttt{I wish I was Laius' son. Then I'd be a real king,} to which I reply,
\texttt{Yeah, too bad, huh? If only.} If the player does know that Oedipus is the son of
Laius, then they may instead say, \texttt{But wait, I am the son of Laius!} to which I
reply, \texttt{Isn't that ironic?}

\paragraph*{The plot of \thegame{}} will be expanded in scope relative to \theplay{}, in
an effort to expose more of Oedipus' backstory and character. This feels like an obvious
choice, as narrative prose is not subject to the same forces which incentivize drama
towards smaller scopes and shorter plots. \footnote{I generally think Aristotle is right
  that tragedies (and all plays) should be small in scope, but that's neither here nor
  there.} In service of allowing the player to understand and inhabit Oedipus, \thegame{}
will follow him from his childhood in Corinth. The rough outline of the scenes I have
planned are:

\begin{enumerate}
\item In Corinth, a short scene of Oedipus' childhood will introduce his disability, his
  related insecurity, and his temper via a schoolyard scuffle.
\item As a young adult, a tutorial set in Polybus' court will introduce the game's
  mechanics; namely facing dilemmas, evaluating options, making choices, and balancing
  spending \selfcontrol{}.
\item At a banquet at Corinth, Oedipus will overhear a rumor about his prophecy from a
  drunk guest. The player will then choose between spending \selfcontrol{} to ignore the
  rumor and stay in Corinth, or running away to Delphi to confirm the rumor. If the player
  opts to stay, the game will end with some lines about how, on the one hand, Oedipus has
  escaped his fate, but on the other hand, there isn't any story to tell. The player will
  be encouraged to go back and replay, choosing to leave for Delphi.
\item At Delphi, Oedipus will here the prophecy. Unlike the rest of the game's text, the
  oracle will speak in verse, possibly using lines taken from Gilbert Murray's 1911
  translation in rhyming verse. \footnote{The oracle does not deliver the prophecy in
    \theplay{}, but it contains enough attestations of the prophecy that I think I could
    piece something together.} No opportunity for player interaction will be provided
  during this scene.
\item At the foot of Mount Parnassus, Oedipus will be insulted by a carriage-driver, who
  will whip at Oedipus' feet to get him to move out of the way. An option will be shown to
  have Oedipus keep his cool, with a description that calls out how unwise it is to get
  into fights when one is trying to avoid a prophecy about committing murder, but it will
  be programmed to require one more \selfcontrol{} than the player has at the moment,
  regardless of how much they have spent in preceding scenes. The only option actually
  available to the player will be described simply as \texttt{Lose your cool}, and will
  describe Oedipus' slaughtering the entire carriage procession.
\item After Oedipus comes to his senses, the player will be offered a chance, at the cost
  of \selfcontrol{}, to return to Corinth in spite of the prophecy. This will result in an
  ending similar to the one if the player stays in Corinth after hearing the
  rumor. Otherwise, the player will continue on to Thebes.
\item The player will confront the Sphinx at Thebes. Dialogue attributed to me will call
  out the Sphinx's tie to Egypt, and invite the player to reflect on it. The player will
  be shown a text box and asked to type the answer to the Sphix's riddle,
  \texttt{man}. After each incorrect guess, dialogue attributed to me will offer
  increasingly obvious hints, including \texttt{Just go Google \enquote{riddle of the
      sphinx,} okay?}
\item Joyful citizens of Thebes will crown Oedipus Tyrant of Thebes.
\item A series of vignettes will show a few years of Oedipus' reign as tyrant, the
  citizens' satisfaction with his rule, his marriage to Jocasta and the birth of his
  children.
\item A final vignette will introduce the plague, and Creon will arrive to inform the
  player of the curse and the requirements for dispelling it.
\item Oedipus' investigation will play out non-linearly, with the player choosing the
  order to investigate various leads. After pursuing a lead, they will be returned to a
  hub scene set at the palace at Thebes, where they can choose another.
\item Once the player has accumulated enough \knowledge{} to solve the mystery
  \footnote{Or, once they have accumulated enough \knowledge{} that they can no longer
    deny the obvious.}, they will choose between killing themselves, or, at the cost of
  \selfcontrol{}, putting out their eyes and leaving the city. Players who spend too much
  \selfcontrol{} during the investigation will have no option but suicide. If the player
  chooses suicide, the game will end immediately, with no resolution. Otherwise,
  \thegame{} will end the same way as \theplay{}.
\end{enumerate}

